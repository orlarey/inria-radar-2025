\subsection{The Faust Programming Language and its Ecosystem} {\bf Participants: \florent, \stephane, \romain,  \yann, \tanguy, \christine}


Audio signal processing is an applied field where success is ultimately determined by the human ear, requiring advanced tools to prototype and implement algorithms rapidly and efficiently. The \F{} programming language and environment, developed at \gramecncm in 2002~\cite{orlarey2002}, represented a significant development by enabling researchers and developers to prototype and deploy audio processing algorithms more efficiently. At the time, \F{} was the first fully compiled audio programming language, which played an important role in making the field of real-time embedded audio systems more accessible.  

Since its inception, \F{} has gained international recognition and is widely used in both academic and professional contexts. It has been adopted for teaching advanced topics such as signal processing and physical interaction design at Stanford University and for developing audio plugins~\cite{faustLibraries2019}.  

Today, the Faust research project, conducted by \projectname, focuses on three interrelated areas:  
\begin{itemize}
    \item \textbf{The Programming Language}: Developing a high-level language for sound synthesis and signal processing that is accessible to non-computer scientists.
    \item \textbf{The Compiler and Compilation Techniques}: Producing tools to automatically generate highly optimized code, comparable in efficiency to code written by experienced C programmers.
    \item \textbf{The Ecosystem}: Expanding and maintaining architecture files (which allow the same Faust code to run on over twenty platforms), development tools, libraries, and documentation.
\end{itemize}  

The following sections present the work carried out during the period in each of these areas.

\paragraph{The \F{} programming language}

\F{} is a synchronous functional programming language inspired by lambda-calculus, 
combinatory logic and John Backus’ FP.
It's semantics is centered on the concept of `audio
circuits,' where each function or expression represents a component
within a signal processing chain. Initially, the language served
primarily to assemble these circuits, with computations processed at
every sample to ensure sample-accurate processing. This approach, while
effective, required continuous computation even when it wasn't strictly
necessary. During the period, however, \F{} has evolved beyond circuit
assembly: new primitive operations, such as ondemand and Widget
Modulation, now allow users to transform circuits and selectively
control when computations occur. These transformative features introduce
a new level of control over signal flow, computation timing, and
parameter modulation, making \F{} an even more flexible and powerful
language for developing complex, interactive audio applications.

\subparagraph{Widget Modulation}

Building on this evolution, Widget Modulation introduces an innovative
extension to \F{} that allows developers to modulate the values of user
interface elements, or `widgets,' within an audio circuit without
modifying the original code~\cite{orlarey:hal-04762253}. Inspired by the flexibility of modular
synthesizers, this feature enables voltage-control-like modulation by
specifying a modulation circuit that can dynamically adjust parameters.
This post-development control promotes reuse and customization,
enhancing \F{}'s modularity by allowing users to transform interface
behaviors without altering underlying circuits. With various modulation
techniques---additive, multiplicative, or widget replacement---Widget
Modulation expands \F{}'s versatility, empowering developers to design
intricate, interactive audio interfaces with ease.

To achieve this, the \F{} compiler first evaluates the original circuit
in a `normal form,' a flattened representation that standardizes the
circuit's structure. Next, it identifies every point in the circuit
where the target widget appears, systematically inserting modulation
circuitry at each relevant location. This process effectively generates
a new, fully modulated circuit that incorporates the specified
modulation behavior, seamlessly embedding dynamic control within the
audio flow.

\subparagraph{Ondemand}

The ondemand extension in \F{} introduces a powerful mechanism to
control when computations occur within an audio circuit, adding
flexibility and efficiency to audio processing. Traditionally, \F{}
computes signals at every sample, but with ondemand, computations are
triggered only when a clock signal is active, allowing selective
processing based on demand. This feature is implemented by adding a
clock input to the circuit: when the clock is high (1), computations
proceed; when it is low (0), they are skipped.

Internally, ondemand leverages downsampling and upsampling techniques to
manage these on-off states, synchronizing computations with the clock
signal to maintain timing consistency. The \F{} compiler transforms
circuits into `gated' versions, where computations activate only as
needed, significantly optimizing performance by reducing unnecessary
calculations. This approach is particularly advantageous for large,
complex synthesizers---common in commercial applications---that offer
extensive functionalities users can activate or deactivate at will. In
such cases, ondemand ensures that unused features don't consume CPU or
energy resources, making \F{} highly efficient for applications that
demand responsive, resource-conscious performance. Inspired by
demand-rate operations in SuperCollider, ondemand maintains \F{}'s
streamlined, sample-based semantics while enabling efficient,
clock-driven processing within a continuous framework.

\subparagraph{Automatic differentiation}

Differentiable Digital Signal Processing (DDSP) \cite{engel2020ddsp} is the application of differentiable programming to audio computations. Coupled with gradient-based optimization methods, differentiable signal processors are central to a variety of audio problems and can be incorporated into machine learning architectures.

In order to use \F{} for this type of application, we are developing two complementary approaches. The first is to express directly in Faust, using the language's pattern matching rules, the transformation of a Faust circuit into its differentiated version. This approach is presented in \cite{rushton:hal-04849619}. The second approach involves introducing a new differentiation construct into the language itself and implementing this transformation directly in the \F{} compiler. This is the subject of Benjamin Quiedeville's thesis, which began in September 2024.


\paragraph{The \F{} compiler}

To support \F{}'s evolving functionality and optimize performance
across a wider range of platforms, recent advancements in the \F{}
compiler---driven in part by the Syfala project---have introduced
innovative compilation techniques that now target not only CPUs but also
FPGAs, enabling efficient code generation and dynamic processing
capabilities on diverse hardware architectures.

\subparagraph{Fixed-point compilation}

Modern digital signal processing often relies on floating-point
arithmetic for its simplicity and ease of use. However, fixed-point
number formats can significantly reduce resource usage and improve
execution times, making them particularly beneficial for FPGA
implementations. The challenge with fixed-point formats lies in the
manual effort required to specify formats at each computation step, a
complex and time-consuming task for programmers. Recent advancements in
the \F{} compiler address this by introducing an automatic fixed-point
format determination, a prototype implementation that leverages existing
methodologies to calculate optimal fixed-point formats automatically.
This new feature reduces programming overhead while maximizing the
efficiency of \F{}-generated code on hardware like FPGAs.

\subparagraph{FIR/IIR reconstruction}

Faust's design philosophy deliberately excludes high-level signal processing functions that can be expressed using existing lower-level primitives. Consequently, the language does not provide FIR and IIR filters as direct primitives, but instead includes them in its standard libraries.

But, from a code generation perspective, having high-level operations like IIR and FIR filters can be beneficial. For example, this allows loops to appear in the generated code, which is particularly interesting when using high-level synthesis tools like Vivado HLS. Therefore, we introduced new compilation options that automatically reconstruct FIR and IIR filters in the compiler's internal representations. While these options may not necessarily improve CPU performance, they are clearly very efficient on FPGAs in terms of latency and resource trade-offs.

\subparagraph{Delay lines}

We have developed new strategies for implementing delay lines, focusing on adapting implementation techniques based on delay length and hardware characteristics. The goal is to optimize memory usage, computational efficiency, and resource allocation across different platforms like CPUs and FPGAs. Our new approach dynamically selects implementation methods, from simple array storage for short delays, to ring buffer techniques for long delays - to maximize performance while minimizing resource overhead.

\subparagraph{Instruction scheduling}

At the intermediate signal level, each Faust program is represented as a directed graph (DG), where nodes correspond to individual signals, and edges represent their dependencies. Multiple topological orderings can satisfy the dependency constraints, but due to memory cache and data locality considerations, not all orderings are equally efficient.

We have implemented several new scheduling strategies. Tests reveal that no single strategy consistently outperforms the others. However, the performance impact of the chosen strategy is often highly significant, underscoring the importance of allowing users to select their preferred scheduling method.

\subparagraph{Backends}

The \F{} compilation pipeline incorporates an internal imperative
representation, encompassing memory access operations, mathematical
functions, and control structures, which enables the generation of code
in an ever-growing array of target languages through backends.
Initially, C and C++ backends were developed, followed by an LLVM IR
backend that, when used with a JIT compiler, enabled a fully embeddable,
dynamic compilation chain known as libfaust. The addition of a
WebAssembly backend, utilized in the Web version of libfaust (compiled
from C++ to WebAssembly and JavaScript via the Emscripten compiler),
unlocks an entirely new category of use cases.

Four new backends have been recently developed:

\begin{itemize}
\item
  the JAX backend introduces new opportunities for \F{}, especially in
  the machine learning domain, by leveraging JAX's high-performance
  numerical capabilities and automatic differentiation. JAX enhances
  NumPy by enabling gradient computation and supports JIT compilation
  for optimized performance. An external contributor in collaboration
  with GRAME, has developed a JAX backend for \F{}, allowing the
  creation of differentiable \F{} programs within DawDreamer, an
  audio-processing Python framework. Initial experiments include Flax
  examples with a learnable low-pass filter, differentiable
  synthesizers, and a parametric equalizer in a QDax environment.
\item
  developed by Cockos, the creators of Reaper, JSFX is a scripting
  language which allows users to extend the capabilities of the DAW with
  custom audio and MIDI processing scripts adapted to their specific
  needs. In collaboration with GRAME, an external contributor has
  developed a backend enabling the creation of synthesizers and effects
  with MIDI control, as well as polyphonic MIDI-controllable audio
  plugins generated in JSFX to be loaded and executed in Reaper
  seamlessly.
\item
  Cmajor is a C like procedural high-performance language specifically
  designed for audio processing providing a runtime with dynamic LLVM
  JIT based compilation. The language supports a signal flow through a
  graph structure with processor nodes containing implementations of
  specific DSP building blocks. A Cmajor backend has been written to
  generate a processor from a \F{} DSP program. Parameters such as
  sliders, buttons, and bar graphs correspond to Cmajor's concept of
  input and output events.
\item
  RNBO is a library and toolchain that exports Max-like patches as
  portable code, compilable to targets such as VST, Max externals, or
  Raspberry Pi. \F{} programs can be compiled to RNBO's internal
  codebox\textasciitilde{} language, generating sections for parameter
  definitions, DSP state, initialization, and compute functions. The
  \texttt{faust2rnbo} tool converts \F{} DSP programs into RNBO patches
  containing a rnbo\textasciitilde{} object with embedded codebox code
  as a subpatch, automatically setting up audio I/O and parameters. The
  Python-based script, utilizing py2max, supports DSP and polyphonic
  MIDI-controllable programs, with a tutorial available.
\end{itemize}


\paragraph{The \F{} ecosystem}

The \F{} ecosystem includes the compiler which generates optimized code
formats for various platforms. It offers extensive libraries for audio
processing, synthesis, and effects, alongside development tools like the
\F{} Web IDE to write and test code in real time. \F{} architecture
files enable integration into a wide range of audio environments,
supporting plugins, standalone applications, and web deployment.
Activation across these platforms is streamlined with a comprehensive
suite of \texttt{faust2target} scripts, customized for diverse
deployment needs.

\subparagraph{\F{} and the WebAudio platform}

Recent advancements in \F{} language support on the Web platform
include:

\begin{itemize}
\item
  the \F{}Wasm library which offers a high-level API for compiling
  \F{} DSP code into WebAssembly, compatible with both Node.js and web
  browsers. It enables use as WebAudio nodes in a standard WebAudio node
  graph and supports offline rendering. The library also facilitates SVG
  generation from \F{} programs and supports mono and polyphonic nodes
  with MIDI and sensor integration for use on mobile platforms
  (slartphones and tablets). The \F{} IDE, Editor, and Playground have
  been updated to use this new \F{}Wasm package, enhancing usability
  and tools for \F{} development.
\item
  the \texttt{faust-web-component} package which offers two web
  components for embedding interactive \F{} code directly into web
  pages. These components include an editor with a graphical user
  interface, access to controls, and an SVG diagram renderer for
  visualizing DSP structures.
\item
  a standard for Web Audio plugins and DAWs, WAM 2.0, released in 2021,
  provides an SDK, API, plugins, and hosts to support diverse
  development workflows. \F{}-compatible WAM plugins can be created
  using the \F{} IDE with adapted targets.
\end{itemize}

\subparagraph{\F{}2Eurorack}

The integration of \F{} with Eurorack-blocks has been done.
Eurorack-blocks is a framework capable of programmatically generating
Eurorack digital modules' hardware and firmware files for manufacturing,
and providing a virtual environment for early-stage design, development,
testing and debugging iterations on a desktop computer. It presents a
method to statically bind the inherently nested structure of a \F{} DSP
program with the flat namespace and different types of the ERBUI and
ERBB languages, which are Domain Specific Languages to describe the
Eurorack module UI, module DSP file and associated audio data,
respectively. An implementation is demonstrated, taking into
consideration the specific memory model of the hardware embedded
platform, as well as the meta-programming technique used to minimize
computations done at run time by relocating them at build time.
